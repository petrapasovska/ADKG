\documentclass[a4paper, 12pt]{article}
\usepackage[total={17cm,25cm}, top=2.5cm, left=2.5cm, right=2.5cm,  includefoot]{geometry}
\usepackage[utf8]{inputenc}
\usepackage{array}
\usepackage{multirow}
\usepackage{hhline}
\usepackage{gensymb}
\usepackage{graphicx}
\graphicspath{ {} }
\usepackage[czech]{babel}
\usepackage{enumitem}
\usepackage{pdfpages}
\usepackage{amsmath}
\usepackage{verbatim}
\usepackage{listings}
\usepackage{hyperref}
\usepackage{amssymb}


\pagestyle{empty} % vypne číslování stránek




\usepackage[OT2,OT1]{fontenc}
\newcommand\cyr
{
\renewcommand\rmdefault{wncyr}
\renewcommand\sfdefault{wncyss}
\renewcommand\encodingdefault{OT2}
\normalfont
\selectfont
}
\DeclareTextFontCommand{\textcyr}{\cyr}
\def\cprime{\char"7E }
\def\cdprime{\char"7F }
\def\eoborotnoye{\char’013}
\def\Eoborotnoye{\char’003}


\begin{document}



\begin{titlepage}
\begin{center}
\noindent
\Large \textbf{České vysoké učení technické v Praze }\\ Fakulta stavební
\vspace{5cm}

\huge

%vložení loga cvut
\begin{figure}[h!]
	\centering
	\includegraphics[width=7cm]{logo.png}
\end{figure}

\vspace{0.5cm}

Algoritmy v digitální kartografii \\

\vspace{3cm}

\Huge  
Digitální model terénu a jeho analýzy\\

\vspace{2cm}

\Large
Bc. Petra Pasovská \\
Bc. David Zahradník \\

\end{center}

\end{titlepage}




\pagestyle{plain}     % zapne obyčejné číslování
\setcounter{page}{1}  % nastaví čítač stránek znovu od jedné

\tableofcontents
\newpage

\section{Zadání}
Níže uvedené zadání je kopie ze stránek předmětu. 

\begin{figure}[h!]
	\centering
	\includegraphics[clip, trim=0cm 10cm 0cm 3cm, width=1.2\textwidth]{zadani.pdf}
\end{figure}

\subsection{Údaje o bonusových úlohách}
Nebyly vytvořeny žádné bonusové úlohy.


\clearpage

\section{Popis a rozbor problému}
Hlavním cílem této úlohy je tvorba aplikace, která nad výškopisnými body vytvoří Delaunay triangulaci, vrstevnice a vypočte sklon a expozici k světovým stranám.\\
\\
Obecně triangulační algoritmy jsou nejvíce zkoumané algoritmy digitální kartografie v dnešní době. Slouží k různým účelům např. tvorba digitálního modelu terénu, plánování pohybu robotů, detekce otisků prstů.  [Zdroj: 1]
\\


\section{Popis použitých algoritmů}
Existuje několik způsobů jak vytvořit triangulaci s různými kritérii. V této úloze jsme ze zabývali Delaunay triangulací pomocí metody inkrementální konstrukce.

\subsection{Delaunay triangulací}
Delaunay triangulace je nejčastěji používanou trianglucí při tvorbě digitálního modelu terénu. Delaunay triagulaci lze provádět v rovině i v prostoru.
\\
\subsubsection{Vlastnosti}
Uvnitř opsané kružnice libovolného trojúhelníku triangulace neleží žádný jiný bod.
\\
Maximalizuje minimální úhel, avšak neminimalizuje maximální úhle v trojúhelníku.
\\
Vůči kritériu minimálního úhlu je lokálně i globálně optimální.
\\
Triangulace je jednoznačná pokud čtyři body neleží na kružnici.
\\

Triangulace byla realizována pomocí metody inkrementální konstrukce. Tento algoritmus je založen na postupném hledání bodu, který k bodům hrany tvoří minimální opsanou kružnici. Každá hrana je orientovaná a bod se hledá pouze v její levé polorovině.\\
Je-li nalezen bod s výše uvedeným kritériem vytvoří se dvě nové hrany, které jsou přidány do triangulace. Nenalezne-li se daný bod, prohodí se orientace hrany a hledání pokračuje.\\
Hrany, které nebyly zlegalizovány (nebyl k ní ještě nalezen třetí bod), jsou ukládány do struktury Active Edge List (AEL). Pokud k dané hraně byl nalezen tří bod, hrana se ze struktury odstraní. Než je hrana vložena do struktury, kontroluje se zda hrana už ve struktuře není s opačnou orientací, pokud je, hrana se nevloží. Algoritmus běží dokud struktura AEL není prázdná.

\subsubsection{Implementace metody}
\begin{enumerate}
\item Nalezení pivota q a k němu nejbližší bod:  $ q = min(y_i) $ 
\item Hledání prvního Delaunay bodu.
\item Vytvoření prvního Delaunay trojúhelníku.
\item Hrany trojúhelníka se uloží do triangulace a do AEL.
\item Dokud není AEL prázdný proveď.
\subitem Hledej Delaunay bod k hraně z AEL.
\subitem Pokud Delaunay bod existuje.
\subsubitem Přidej nové hrany do DT.
\subsubitem Pokud nová hrana není v AEL přidej.
\end{enumerate}

\subsection{Vrstevnice}
V úloze byly vrstevnice konstruovány lineární interpolací. U lineární interpolace je rozestup vrstevnice mezi dvěma body konstantní, tedy i spád. Při konstrukci vrstevnic hledáme průsečnici vodorovné roviny o výšce Z a rovinu trojúhelníka triangulace.\\

\subsubsection{Implementace metody}
\begin{enumerate}
\item Pro všechny hrany triangulace:
\subitem Otestuj zda hrana protíná vodorovnou rovinu o výšce Z:
\subitem Pokud protíná spočti polohové souřadnice.
\subitem Vytvoř hranu tvořící vrstevnici.

\end{enumerate}

\subsection{Sklon terénu}
Skon terénu je definován jako úhle mezi normálovým vektorem (0,0,1) a normálovým vektorem roviny trojúhelníku.

\subsubsection{Implementace metody}
\begin{enumerate}
\item Pro všechny trojúhelníky triangulace:
\subitem Vypočti sklon:
\end{enumerate}

\subsection{Expozice terénu}
Expozice terénu je definována jako azimut k průmětu normálového vektoru roviny trojúhelníka do roviny x,y.

\subsubsection{Implementace metody}
\begin{enumerate}
\item Pro všechny trojúhelníky triangulace:
\subitem Vypočti expozici:
\end{enumerate}

\section{Informace o bonusových úlohách}


\section{Vstupní data}
\\

\section{Výstupní data}
\\


\clearpage
\section{Aplikace}
\begin{figure}[h!]
	\centering
	\includegraphics[width=10cm]{vstup.jpg}
	\caption{Zobrazené okno po spuštění aplikace}
\end{figure}

\begin{figure}[h!]
	\centering
	\includegraphics[width=10cm]{circle.jpg}
	\caption{Vygenerované body v kruhu}
\end{figure}

\begin{figure}[h!]
	\centering
	\includegraphics[width=10cm]{square.jpg}
	\caption{Vygenerované body ve čtverci}
\end{figure}

\begin{figure}[h!]
	\centering
	\includegraphics[width=10cm]{ellipse.jpg}
	\caption{Vygenerované body v elipse}
\end{figure}

\begin{figure}[h!]
	\centering
	\includegraphics[width=10cm]{grid.jpg}
	\caption{Vygenerované body v pravidelné mřížce}
\end{figure}

\begin{figure}[h!]
	\centering
	\includegraphics[width=10cm]{random.jpg}
	\caption{Náhodně vygenerované body}
\end{figure}

\begin{figure}[h!]
	\centering
	\includegraphics[width=10cm]{star_shaped.jpg}
	\caption{Vygenerované body ve tvaru Star Shaped}
\end{figure}

\begin{figure}[h!]
	\centering
	\includegraphics[width=10cm]{warning_grid.jpg}
	\caption{Varování na redukci vloženého počtu bodů při tvorbě gridu }
\end{figure}

\begin{figure}[h!]
	\centering
	\includegraphics[width=10cm]{warning_square.jpg}
	\caption{Varování při neplatném vstupu při tvorbě čtverce}
\end{figure}

\begin{figure}[h!]
	\centering
	\includegraphics[width=10cm]{warning_square2.jpg}
	\caption{Varování na redukci vloženého počtu bodů při tvorbě čtverce}
\end{figure}

\begin{figure}[h!]
	\centering
	\includegraphics[width=10cm]{hint.jpg}
	\caption{Nápověda, pokud si uživatel nebude jistý, jak v aplikaci postupovat}
\end{figure}

\begin{figure}[h!]
	\centering
	\includegraphics[width=10cm]{tooltip.jpg}
	\caption{Po přiložení kurzoru se zobrazí informace o objektu}
\end{figure}

\begin{figure}[h!]
	\centering
	\includegraphics[width=10cm]{close.jpg}
	\caption{Vyskakovací okno při ukončování aplikace}
\end{figure}

\begin{figure}[h!]
	\centering
	\includegraphics[width=10cm]{jarvis_ch.jpg}
	\caption{Tvorba konvexní obálky metodou Jarvis Scan}
\end{figure}

\begin{figure}[h!]
	\centering
	\includegraphics[width=10cm]{sweepline_ch.jpg}
	\caption{Tvorba konvexní obálky metodou Sweep Line}
\end{figure}

\begin{figure}[h!]
	\centering
	\includegraphics[width=10cm]{qhull_ch.jpg}
	\caption{Tvorba konvexní obálky metodou Quick Hull}
\end{figure}

\begin{figure}[h!]
	\centering
	\includegraphics[width=10cm]{grahamscan_ch.jpg}
	\caption{Tvorba konvexní obálky metodou Graham Scan}
\end{figure}

\begin{figure}[h!]
	\centering
	\includegraphics[width=10cm]{minBound.jpg}
	\caption{Tvorba minimálního ohraničujícího obdélníku}
\end{figure}

\clearpage

\section{Dokumentace}
\subsection{Třídy}
\subsubsection{Algorithms}
Třída Algorithms obsahuje několik metod. Metody jsou určeny pro výpočty použitých algoritmů.
\\

\textbf{double distance(QPoint3D p1, QPoint3D p2)}\\
Metoda, jejíž návratová hodnota je typu double, vrací velikost spojnice mezi dvěma body.
\\

\textbf{TPosition getPointLinePosition(QPoint \&q, QPoint \&a, QPoint \&b)}\\
Tato metoda slouží k určení pozice bodu q vůči linii tvořené body a, b. Výstupem metody je LEFT, RIGHT nebo ON.\\

\textbf{double getCircleRadius(QPoint3D &p1, QPoint3D &p2, QPoint3D &p3, QPoint3D &c)}\\
Metoda jejíž návratová hodnota je typu double, vrací velikost poloměru kružnice tvořené třemi vstupními body.\\

\textbf{int getNearestPoint(QPoint3D &p, std::vector<QPoint3D> &points)}\\
Tato metoda slouží k nalezení indexu nejbližšího bodu k bodu p.\\

\textbf{int getDelaunayPoint(QPoint3D &s, QPoint3D &e, std::vector<QPoint3D> &points)}\\
Tato metoda slouží k nalezení indexu bodu, který splňuje Delaunay vlastnosti.\\

\textbf{std::vector<Edge> DT(std::vector<QPoint3D> &points)}\\
Metoda vytváří nad vektorem bodů Delaunay triangulaci, které se reprezentuje jako vektor hran.\\

\textbf{QPoint3D getContourPoint(QPoint3D &p1, QPoint3D &p2, double z)}\\
Metoda vypočte průsečík hrany, tvořenou 3D body p1 a p2, a rovinou definovanou z souřadnicí.\\

\textbf{std::vector<Edge> createContours(std::vector<Edge> &dt, double z_min, double z_max, double dz)}\\
Metoda z triangulace dt v zadaném intervalu v ose Z <z_min ; z_max> s intervalem vrstevnic dz vrátí vektor hran definující vrstevnice.\\

\textbf{double getSlope(QPoint3D &p1, QPoint3D &p2, QPoint3D &p3)}\\
Tato metoda slouží k vypočetní hodnoty sklonu trojúhelníku definovanému 3D body p1, p2 a p3.\\

\textbf{double getAspect(QPoint3D &p1, QPoint3D &p2, QPoint3D &p3)}\\
Tato metoda slouží k vypočetní hodnoty expozice trojúhelníku definovanému 3D body p1, p2 a p3.\\

\textbf{std::vector<Triangle> analyzeDTM(std::vector<Edge> &dt)}\\
\\

\textbf{std::vector<QPoint3D> generateHill()}\\
\\

\textbf{std::vector<QPoint3D> generateValley())}\\
\\

\textbf{std::vector<QPoint3D> generateMountains()}\\
\\

\textbf{std::vector<QPoint3D> generateRest())}\\
\\

\subsubsection{Draw}
Třída Draw obsahuje několik metod. Metody jsou určeny pro generování a vykreslování proměných.
\\

\textbf{void paintEvent(QPaintEvent *e)}\\
Metoda slouží k vykreslení vytvořených, generovaných bodů a zobrazení výsledků použitých algoritmů.
\\

\textbf{void mousePressEvent(QMouseEvent *e)}\\
Metoda uloží bod se souřadnicemi místa kliknutí v zobrazovacím okně.
\\

\textbf{void clearDT()}\\
Metoda slouží k vymazání proměnných a k překreslení
\\

\textbf{void clearPoints()}\\
Metoda slouží k vymazání bodů.
\\

\textbf{void setPoints(std::vector<QPoint3D> points_)}\\
Metoda slouží pro převod bodů do vykreslovacího okna.\\

\textbf{std::vector<QPoint3D> & getPoints()}\\
Metoda slouží pro převod bodů z vykreslovacího okna.\\

\textbf{std::vector<Edge> & getDT()}\\
Metoda slouží pro převod Delaunay triangulace z vykreslovacího okna.\\

\textbf{void setDT(std::vector<Edge> &dt_)}\\
Metoda slouží pro převod Delaunay triangulace do vykreslovacího okna.\\

\textbf{void setContours(std::vector<Edge> &contours_)}\\
Metoda slouží pro převod vrtevnic do vykreslovacího okna.\\

\textbf{void setDTM(std::vector<Triangle> &dtm_)}\\
\\

\subsubsection{SortByXAsc}
Třída SortByXAsc slouží k porovnání souřadnic v ose x.\\


\textbf{bool operator()(QPoint \&p1, QPoint \&p2)}\\
Přetížený operátor () vrátí bod s větší souřadnicí x z dvojice bodů.\\

\subsubsection{Edge}
\textbf{Edge(QPoint3D &start, QPoint3D &end)}\\
Třída Edge je konstruována ze dvou 3D bodů, počátek a konec hrany. Třída slouží k uložení hrany triangulace a nebo vrstevnic.\\
    
\textbf{QPoint3D & getS()}\\
Metoda vrátí počáteční bod hrany.\\

\textbf{QPoint3D & getE()}\\
Metoda vrátí koncový bod hrany.\\

\textbf{void switchOr()}\\
Metoda prohodí orientaci hrany.\\

\subsubsection{QPoint3D}
\textbf{QPoint3D(double x, double y, double z_)}\\
Třída QPoint3D je odvozena z třídy QPointF a složí k uložení bodu s informací o výšce.\\

\textbf{double getZ()}\\
Metoda vrátí výšce bodu.\\

\textbf{void setZ(double z_)}\\
Metoda nastaví výšku bodu.\\

\subsubsection{Triangle}
\textbf{Triangle(QPoint3D &p1_, QPoint3D &p2_, QPoint3D &p3_, double slope_, double aspect_))}\\
Třída QPoint3D složí k uložení trojúhelníku definovaného boy p1, p2, p3 a jeho informaci o sklonu a expozici.\\

\textbf{ QPoint3D getP1()}\\
Metoda vrátí první bod trojúhelníku.\\

\textbf{ QPoint3D getP2()}\\
Metoda vrátí druhý bod trojúhelníku.\\

\textbf{ QPoint3D getP3()}\\
Metoda vrátí třetí bod trojúhelníku.\\

\textbf{double getSlope()}\\
Metoda vrátí sklon trojúhelníku.\\

\textbf{double getAspect()}\\
Metoda vrátí expozici trojúhelníku.\\

\subsubsection{Widget}

\textbf{void on\_pushButton\_clicked()}\\
Při stisknutí tlačítka Denaulay se zavolá metoda třídy Algorithms DT a výsledek se zobrazí v okně.
\\

\textbf{void on\_pushButton\_3\_clicked()}\\
Při stisknutí tlačítka Clear se zavolá metoda třídy Draw clearDT.
\\

\textbf{void on\_pushButton\_2\_clicked()}\\
Při stisknutí tlačítka Create Contours se zavolá metoda třídy Algorithms createContours a výsledek se zobrazí v okně.
\\

\textbf{void on\_pushButton\_4\_clicked()}\\
Při stisknutí tlačítka AnalyzeDTM se zavolá metoda třídy Algorithms analyzeDTM a výsledek se zobrazí v okně.
\\

\textbf{void on\_pushButton\_5\_clicked()}\\
Při stisknutí tlačítka Generate a výběru z comboboxu se vygenerují body terénu.
\\

\section{Testování}


\clearpage
\section{Závěr}
V rámci úlohy byla vytvořena aplikace, která je schopna na vygenerovaných bodech zkonstruovat konvexní obálku. Zároveň je v rámci aplikace možné náhodně vygenerovat téměř libovolné množství bodů do určitého tvaru. Na výběr je poměrně široké spektrum běžných tvarů, což jistě uživatel ocení.\\
\\
Součástí úlohy bylo i testování doby výpočtu jednotlivých konvexních obálek. Testování bylo prováděno nad striktně konvexními obálkami, samotný výpočet je tedy zatížen ještě o vyhodnocení, zda následující tři body neleží na stejné přímce. Výsledné doby dopadly dle očekávání. Algoritmus Jarvis Scan měl znatelně nejhorší čas. Naopak metoda Graham Scan se jeví jako nejlepší řešení.

\section{Náměty na vylepšení}
\subsection{Graham Scan}
Pro algoritmus Graham Scan byl nejprve napsán kod, který byl velmi náročný a aplikace nebyla funkční na větším množství dat. Z toho důvodu byla napsána nová metoda, která je již funkční a při testování dosahovala nejlepších možných výsledků. Z toho důvodu je v kódu pojmenována jako GrahamScanNew. Původní funkce byla v kódu ponechána pro případnou doeditaci a její zprovoznění.

\subsection{Generování množin bodů}
Při tvorbě generování gridu napadla autory myšlenka, že by se body generovaly v některém ze známých kartografických zobrazení. Z bakalářského studia znají velké množství zobrazovacích rovnic, pomocí nichž by se body zobrazovaly. Zajímavé rozmístění by měla zejména kuželová zobrazení. Při zobrazení bodů v rámci některého kartografického zobrazení by výsledná konvexní obálka představovala celý svět, případně polokouli, záleželo by na typu zvoleného zobrazení.

\subsection{Testování algoritmů}
V rámci této úlohy bylo provedeno několik testování nad odlišným množstvím bodů a na jejich odlišném zobrazení. Časově bylo testování poměrně náročné, za úvahu jistě stojí popřemýšlení, zda by nebylo výhodnější vytvořit takovou funkci, která by pro předem definované tvary a množství bodů sama provedla testování. Zároveň obsahuje platforma Qt i knihovnu grafů, časová úspora by tedy byla velmi výrazná.


\subsection{Tvorba tabulek v LaTeXu}
V rámci dokumentace bylo i prezentování výsledných měření prostřednictvým tabulek. V původních tabulkách byl ještě sloupec, který označoval typy množin (kruh, grid, random). Bohužel s tímto sloupcem byla tabulka oříznutá a přes veškerou snahu se autorům nepodařilo nalézt řešení, jakým způsobem tabulku posunout vlevo. Z toho důvodu byl tento sloupec odstraněn a tabulky byly popsány v záhlaví podkapitoly. S tímto řešením nejsou autoři zcela spokojeni, tabulky nejsou dostatečně přehledné. 

\subsection{Minimum Area Enclosing Box}
Při výpočtu minimálního ohraničujícího obdélníku autorům z neznámé příčiny nefungovala transformace, ač se to snažili vyřešit. Po konzultaci byla část pro transformaci vypůjčena od kolegy, kterému fungovala bez problémů. Vypůjčená část je v kódu vyznačena. 





\clearpage
\section{Reference}

\begin{enumerate}
\item  MARTÍNEK, Petr. Konvexní obálka rozsáhlé množiny bodů v E\^d [online][cit. 31.10.2018]. \\
Dostupné z: http://graphics.zcu.cz/files/86\_BP\_2010\_Martinek\_Petr.pdf  \\

\item Convex Hulls: Explained. [online][cit. 31.10.2018]\\
Dostupné z: https://medium.com/@harshitsikchi/convex-hulls-explained-baab662c4e94\\

\item Convex-hull mass estimates of the dodo (Raphus cucullatus). [online][cit. 31.10.2018]\\
Dostupné z: https://www.semanticscholar.org/paper/Convex-hull-mass-estimates-of-the-dodo-(Raphus-of-a-Brassey-O\'Mahoney/12e07d3b712561cad16501ac8096120e14901eb8

\item GeeksforGeeks: Convex Hull - Set 1 (Jarvis's Algorithm or Wrapping. [online][cit. 5.11.2018]\\
Dostupné z: https://www.geeksforgeeks.org/convex-hull-set-1-jarviss-algorithm-or-wrapping/

\item  BAYER, Tomáš. Geometrické vyhledávání [online][cit. 5.11.2018]. \\
Dostupné z: https://web.natur.cuni.cz/~bayertom/images/courses/Adk/adk4.pdf  \\

\item GeeksforGeeks: Convex Hull - Set 2 (Graham Scan). [online][cit. 12.11.2018]\\
Dostupné z: https://www.geeksforgeeks.org/convex-hull-set-2-graham-scan/



\end{enumerate}
\end{document}



 