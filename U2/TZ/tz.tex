\documentclass[a4paper, 12pt]{article}
\usepackage[total={17cm,25cm}, top=2.5cm, left=2.5cm, right=2.5cm,  includefoot]{geometry}
\usepackage[utf8]{inputenc}
\usepackage{array}
\usepackage{multirow}
\usepackage{hhline}
\usepackage{gensymb}
\usepackage{graphicx}
\graphicspath{ {} }
\usepackage[czech]{babel}
\usepackage{enumitem}
\usepackage{pdfpages}
\usepackage{amsmath}
\usepackage{verbatim}
\usepackage{listings}
\usepackage{hyperref}
\usepackage{amssymb}


\pagestyle{empty} % vypne číslování stránek




\usepackage[OT2,OT1]{fontenc}
\newcommand\cyr
{
\renewcommand\rmdefault{wncyr}
\renewcommand\sfdefault{wncyss}
\renewcommand\encodingdefault{OT2}
\normalfont
\selectfont
}
\DeclareTextFontCommand{\textcyr}{\cyr}
\def\cprime{\char"7E }
\def\cdprime{\char"7F }
\def\eoborotnoye{\char’013}
\def\Eoborotnoye{\char’003}


\begin{document}



\begin{titlepage}
\begin{center}
\noindent
\Large \textbf{České vysoké učení technické v Praze }\\ Fakulta stavební
\vspace{5cm}

\huge

%vložení loga cvut
\begin{figure}[h!]
	\centering
	\includegraphics[width=7cm]{logo.png}
\end{figure}

\vspace{0.5cm}

Algoritmy v digitální kartografii \\

\vspace{3cm}

\Huge  
Konvexní obálky\\

\vspace{2cm}

\Large
Bc. Petra Pasovská \\
Bc. David Zahradník \\

\end{center}

\end{titlepage}




\pagestyle{plain}     % zapne obyčejné číslování
\setcounter{page}{1}  % nastaví čítač stránek znovu od jedné

\tableofcontents
\newpage

\section{Zadání}
Níže uvedené zadání je kopie ze stránek předmětu. 

\begin{figure}[h!]
	\centering
	\includegraphics[clip, trim=0cm 10cm 0cm 3cm, width=1.2\textwidth]{zadani.pdf}
\end{figure}

\subsection{Údaje o bonusových úlohách}
Z bonusových úloh byl vytvořen algoritmus pro automatické generování množin bodů různých tvarů. Uživatel si může v aplikaci sám zvolit, zda chce vygenerovat kruh, elipsu, čtverec, star-shaped či grid. Byla vložena i možnost náhodného rozmístění bodů v daném zobrazovacím okně. Dále byl vytvořen algoritmus pro výpočet konvexní obálky metodou Graham Scan. V rámci aplikace je i fungující výpočet konstrukce Minimum Area Enclosing Box. V neposlední řadě byly všechny vytvořené konvexní obálky zredukovány tak, aby vytvářely striktně konvexní obálky.


\clearpage

\section{Popis a rozbor problému}
Hlavním cílem této úlohy je tvorba aplikace, která pro vygenerované množství bodů vytvoří konvexní obálku za pomoci různých algoritmů. Pro jednotlivé metody byla zapisována i doba trvání algoritmu. Výsledné časy jsou v závěru následně porovnány.\\
\\
Lze říci, že konvexní obálka množiny M je nejmenší konvexní množina, která množinu M obsahuje. V současné době mají konvexní obálky, v některých literaturách označovány jako konvexní obaly, mnoho využití. Často se využívají jako první odhad tvaru nějakého prostorového jevu, např. detekce kolizí, detekce natočení budov a jejich tvaru v kartografii, analýza shluků atd. [Zdroj: 1]
\\

\begin{figure}[h!]
	\centering
	\includegraphics[width=9cm]{convex_hull.jpg}
	\caption{Ukázka vstupních bodů, kolem nichž je vytvořena konvexní obálka. [zdroj: 2]}
\end{figure}

Konvexní obálky využívá celá řada vědních oborů. Zajímavé bylo využití konvexních obálek v paleontologii, kde za pomoci konvexních obálek jsou vědci schopni určit přibližně tvar těla vyhynulých živočichů, jejichž kosti byly nalezeny. \\

\begin{figure}[h!]
	\centering
	\includegraphics[width=10cm]{paleontology.jpg}
	\caption{Využití konvexních obálek v paleontologii [zdroj: 3]}
\end{figure}


\section{Popis použitých algoritmů}
Existuje několik způsobů jak vytvořit konvexní obálku. V této úloze byly použity pro tovrbu 4 metody - Jarvis Scan, Quick Hull, Sweep Line a Graham Scan.

\subsection{Jarvis Scan}
Tato metoda bývá přirovnávána ke způsobu balení dárků (alternativní název Gift Wrapping Algorithm). Předpokladem pro algoritmus Jarvis Scan je, že 3 body nesmí ležet na jedné přímce. Metoda je poměrně snadná pro zápis, velkou nevýhodou je však časová náročnost O($n^2$), které lze dosáhnout, pokud body z množiny S leží na kružnici. Běžný čas výpočtu bývá O(n*h), kde n je počet vstupních bodů a h je počet bodů, které tvoří obálku. [zdroj: 1]\\
\\
Metoda je pojmenována po R. A. Jarvisu, který ji publikoval v roce 1973. \\
\\
Abychom byli schopni algoritmus sestavit, je potřeba nalézt pivot, označme jej q. Nalezení pivota má časovou náročnost O(n). Pivota nalezneme jako bod s minimální hodnotou souřadnice Y. Následně porovnáváme úhel, který svírá pivot a bod následující a předcházející pivotu, dokud nenalezneme maximální úhel. Když takovýto úhel nalezeneme, je přidán mezi body konvexní obálky. V algoritmu dojde k přeindexování bodů a jsou porovnávány následující body, dokud nově vložený bod není pivot. 

\begin{figure}[h!]
	\centering
	\includegraphics[width=10cm]{jarvis.jpg}
	\caption{Princip Jarvis Scan algoritmu [zdroj: 4]}
\end{figure}

\subsubsection{Problematické situace}
K chybě v algoritmu může dojít v případě, že v množině bodů bude existovat více bodů s minimální Y souřadnicí. V tom případě se jedná o kolineární body. Tento problém lze ošetřit výběřem pivota s minimální Y souřadnicí a maximální X souřadnicí.\\

Ošetření problematické situace:
\begin{enumerate}
\item Nalezení pivota q:  $ q = min(y_i) $ 
\item Při nalezení více bodů s minimální souřadnicí: $max X_{ \forall q \in P}$
\end{enumerate}




\subsubsection{Implementace metody}
\begin{enumerate}
\item Nalezení pivota q:  $ q = min(y_i) $ 
\item Přidej bod q do konvexní obálky:  $ q \rightarrow H  $ 
\item Inicializuj: $p_j = q; p_{j+1} = p_{j-1}$
\item Opakuj, dokud: $ p_{j+1} \ne q $
\subitem Nalezni $p_{j+1}$: $ p_{j+1} = arg  max_{\forall p_i \in P}  \angle (p_{j-1}, p_j, p_i)$
\subitem Přidej $p_{j+1}$: $ p_{j+1} \rightarrow H  $
\subitem Přeindexování bodů: $ p_{j-1} = p_j; p_j = p_{j+1}  $
\end{enumerate}

\subsection{Quick Hull}
Metoda Quick Hull slouží k vytvoření konvexní obálky nad konečným počtem bodů. Využívá techniku "Rozděl a panuj", označovanou anglicky "Divide and Conquer". V této metodě lze najít analogii s QuickSortem, odkud také pochází označení algoritmu. Jedná se o poměrně rychlý algoritmus, časová náročnost je O(n*log(n)), v nejhorším případě je však časová náročnost kvadratická O($n^2$). \\

Pro výpočet metodou Quick Hull je nejprve zapotřebí nalézt extrémní body, v aplikaci byly souřadnice setříděny podle x-ové souřadnice. Bod s nejnižší a nejvyšší hodnotou x-ové souřadnice je vložen do množiny, do které uchováváme body konvexní obálky. Těmito body je vedena pomyslná přímka, která množinu bodů rozdělí na dvě množiny - horní a dolní. V každé polorovině nalezneme nejvzdálenější bod od přímky, tento bod přidáme do množiny bodů patřících do konvexní obálky a vytvoříme přímky tohoto bodu a krajních bodů předešlé přímky. Následně pokračujeme analogicky a nad každou nově vzniklou přímkou nalezneme nejvzdálenější bod.

\begin{figure}[h!]
	\centering
	\includegraphics[width=10cm]{quickhull.jpg}
	\caption{Princip Quick Hull algoritmu [zdroj: 5]}
\end{figure}

\subsubsection{Implementace globální metody}
\begin{enumerate}
\item Vytvoření množiny konvexní obálky, horní a dolní množiny:  $H = 0; S_U = 0; S_L = 0 $ 
\item Nalezení extrémních hodnot:  $ q_1 =  min_{\forall p_i \in S}(x_i); q_3 =  max_{\forall p_i \in S}(x_i) $ 
\item Přidání extrémních bodů do horní a dolní množiny: $S_U \leftarrow q_1; S_U \leftarrow q_3; S_L \leftarrow q_1; S_L \leftarrow q_3 $
\item Pro všechny body množiny: $\forall p_i \in S  $
\subitem Rozhodnutí, zda bod patří do horní množiny: $ if(p_i \in \sigma_l(q_1, q_3)) S_U \leftarrow p_i  $
\subitem V opačném případě: $ S_L \leftarrow p_i  $
\item Přidání krajního bodu do konvexní obálky: $H \leftarrow q_3$
\item Nalezení nejvzdálenějšího bodu $c$ v horní části od přímky, přidání do množiny konvexní obálky a opakování vůči nově vzniklé přímce.
\item Přidání krajního bodu do konvexní obálky: $H \leftarrow q_1$
\item Opakování hledání nejvzdálenějšího bodu v dolní části.
\end{enumerate}



\subsection{Sweep line}
Metoda Sweep line, v češtině označovaná také jako metoda zametací přímky, využívá strategii inkrementální konstrukce. Množinu bodů si v dané dimenzi rozdělíme na zpracovanou a nezpracovanou část. Rozdělovací kritérium je ve většině případů jedna ze souřadnicových os. Je zde tedy nutné body podle této osy seřadit a následně vyhodnocovat každý následující nezpracovaný bod. Body se vyhodnocují v závislosti na jejich poloze vůči tečně. Časová složitost této metody je O(n log(n)). 

\begin{figure}[h!]
	\centering
	\includegraphics[width=10cm]{sweep_line.jpg}
	\caption{Princip algoritmu zametací přímky [zdroj: 5]}
\end{figure}

\subsubsection{Problematické situace}
Problémy v této metodě mohou nastat v případě, že na vstupu budou duplicitní body. Tyto duplicitní body vytvoří singularity, které znemožní správný výpočet algoritmu. Tuto situaci je nutné ošetřit na začátku algoritmu tím, že budou z množiny všech bodů odstraněny duplicitní body.\\

Ošetření problematické situace:
\begin{enumerate}
\item Nalezení duplicitního bodu:  $ if( (points[j].x == points[i].x) \&\& (points[j].y == points[i].y) )$ 
\item Odstranění bodu z množiny: $ Delete$ $ points[i]$
\end{enumerate}


\subsubsection{Implementace metody}
\begin{enumerate}
\item Seřazení bodů množiny podle osy x: $ Sort P_s = sort(P) ~by~ x $ 
\item Vyhodnocení v případě, že bod $p_3$ leží v levé polorovině od přímky $ p_1, p_2$:  \\
$ if (p_3 \in \sigma_L (p_1, p_2)) $ 
\subitem Změna indexů následovníků: $ n[1] = 2; n[2] = 3; n[3] = 1  $
\subitem Změna indexů předchůdců: $ p[1] = 3; p[2] = 1; p[3] = 2  $
\item Vyhodnocení v případě, že bod$ p_3$ leží v levé polorovině od přímky $ p_1, p_2$:  \\
$ if (p_3 \in \sigma_P (p_1, p_2)) $ 
\subitem Změna indexů následovníků: $ n[1] = 3; n[3] = 2; n[2] = 1  $
\subitem Změna indexů předchůdců: $ p[1] = 2; p[3] = 1; p[2] = 3  $
\item Vyhodnocování následujících bodů: $ for~ p_i \in P_S, i \textgreater 3$
\subitem Porovnání hodnoty souřadnice y: $ if (y_i \textgreater y_{i-1})  $
\subsubitem Změna indexů při splnění podmínky: $ p[i] = i-1; n[i] = n[i-1]$
\subsubitem V opačném případě: $ p[i] = p[i-1]; n[i] = i-1$
\subitem Přeindexování následníka předchůdce a předchůdce následníka: \\
$ n[p[i]] = i; p[n[i]] = i  $
\subitem Zhodnocení polohy následníka následníka vůči přímce bodu a následníka: \\
$ while (n[n[i]]) \in \sigma_R (i, n[i])  $
\subsubitem Změna indexů: $ p[n[n[i]]] = i; n[i] = n[n[i]]$
\subitem Zhodnocení polohy předchůdce předchůdce vůči přímce bodu a předchůdce: \\
$ while (p[p[i]]) \in \sigma_L (i, p[i])  $
\subsubitem Změna indexů: $ n[p[p[i]]] = i; p[i] = p[p[i]]$
\end{enumerate}


\subsection{Graham Scan}
Algoritmus Graham Scan slouží k vytvoření konvexní obálky nad konečným množstvím bodů. Metoda je pojmenována po Ronaldu Grahamovi, který ji publikoval v roce 1972. Časová náročnost metody je O(n log(n)). Hlavní myšlenka algoritmu je taková, že každá uspořádaná trojice bodů musí splňovat kritérium levotočivosti (uvažujeme uspořádání hran v CCW orientaci - proti směru hodinových ručiček).\\
\\
Pro výpočet je nejprve nutné body seřadit dle námi zvoleného kritéria, v tomto případě podle souřadnice Y. Bod s nejnižší souřadnicí Y označíme jako pivot. Následně je vypočtena směrnice s osou X vůči pivotu všech ostatních bodů. Body jsou následně podle tohoto úhlu setříděny. Po setřídění je možné přejít k vyhodnocování polohy bodů, kde jsou vždy testovány 2 poslední přidané body v polygonu konvexních obálek a následující setříděný bod.

\begin{figure}[h!]
	\centering
	\includegraphics[width=10cm]{graham_scan.jpg}
	\caption{Princip vyhodnocení polohy bodu v metodě Graham Scan [zdroj: 6]}
\end{figure}

\subsubsection{Problematické situace}
Problém může nastat, pokud nalezneme dva body, které svírají s pivotem a osou X stejný úhel. Tato situace je ošetřena tím způsobem, že při nalezení stejného úhlu je vypočtena vzdálenost pivota a obou bodů. Následně je vzdálenost porovnána a bod s kratší vzdáleností je z množiny bodů odstraněn a do dalšího výpočtu již není uvažován.\\

Tuto problematickou situaci ošetřujeme uvnitř kódu (viz Implementace metody - 3. řádek). 

\subsubsection{Implementace metody}
\begin{enumerate}
\item Nalezení pivota q:  $ q = min_{\forall p_i \in S} (y_i), q \in H $ 
\item Setřídění bodů dle úhlu s osou x: $ {\forall p_i \in S}$ sort by $ \omega_i = \angle (p_i, q, x)$
\item Při nalezení stejného úhlu: $ \omega_k = \omega_l \rightarrow $ delete the closer point 
\item Vložení pivota a prvního bodu do množiny: $ H \leftarrow q; H \leftarrow p_1$
\item Opakuj pro všechna: $ for j < n $
\item Vyhodnocení polohy bodu: if $ p_j $ vpravo od předešlých bodů $ \rightarrow pop S$
\item V opačném případě přidej bod do konvexní obálky: $ p_{j}  \rightarrow H $
\end{enumerate}

\section{Informace o bonusových úlohách}
V podkapitolách jsou blíže popsány bonusové úlohy kromě algoritmu Graham Scan, ten byl již vysvětlen a popsán v rámci použitých algoritmů. Zároveň není blíže popsán stav kolineárních bodů při metodě Jarvis Scan. Tento stav je specifikován v rámci kapitoly zabývající se algoritmem Jarvis Scan v podkapitole Problematické situace.

\subsection{Automatické generování množin bodů}
Pro tvorbu automatického generování množin bodů je možné zvolit útvar, v němž budou generované body vykresleny. V rámci aplikace si může uživatel vybrat mezi kruhem, čtvercem, elipsou, gridem, náhodném rozmístění či star shaped útvarem. Veškeré body jsou odsazeny, aby elipsy, které bod reprezentují, nebyly nějakým způsobem oříznuté.

\subsubsection{Circle}
Pro tvorbu kruhu byl nejprve náhodně vygenerován střed kružnice a její poloměr. V závislosti na počtu bodů byl určen středový úhel po sobě jdoucích bodů. Po znalosti těchto hodnot bylo snadné vygenerovat body na kružnici:\\
$ X = X_0 + r * cos(\phi)$ \\
$ Y = Y_0 + r * sin(\phi) $

\subsubsection{Ellipse}
Elipsa byla generována obdobně jako kruh s tím rozdílem, že nebyl generován poloměr, ale hodnota hlavní poloosy (a) a vedlejší poloosy (b). \\
$ X = X_0 + a * cos(\phi)$ \\
$ Y = Y_0 + b * sin(\phi) $

\subsubsection{Square}
Při tvorbě čtverce je nejprve vygenerováno náhodné umístění levého horního rohu. Poté je náhodně vypočtena délka hrany. Se znalostí délky hrany jsme schopni určit zbylé vrcholy čtverce a zároveň zobrazit body na hranách. Při generování čtverce je vstupní množství zredukováno tak, aby počet bodů odpovídal počtu nutnému pro zobrazení čtverce. Pro představu - z pěti bodů čtverec nesestavíte, tudíž pokud to je nutné, tak dochází k přepočítání počtu bodů. Na tento fakt je uživatel aplikací upozorněn.

\subsubsection{Star Shaped}
Při tvorbě Star Shaped polygonu je automaticky vygenerován střed a poloosy a, b. Pro polygony typu star shaped je typické, že z jeho středu lze vidět do všech vrcholů. Dva po sobě jdoucí body budou vytvořeny s různou vzdáleností od středu, čímž této podmínky docílíme. Středový úhel mezi dvěma následujícími body bude opět určen v závislosti na množství vstupních dat.\\
$ X_i = X_0 + a * cos(\phi)$ \\
$ Y_i = Y_0 + a * sin(\phi) $ \\
$ X_{i+1} = X_0 + b * cos(\phi)$ \\
$ Y_{i+1} = Y_0 + b * sin(\phi) $

\subsubsection{Grid}
Pro tvorbu pravidelné mřížky je zapotřebí, aby vstupní počet bodů byl druhou mocninou nějakého přirozeného čísla. Pokud tato podmínka není splněna, je počet redukován a uživatel je o změně informován. Takovýto případ může nastat například při zadání 18 bodů. Počet je v tomto případě zredukován na 16 a vytvořený grid má velikost 4 x 4. Podobně jako při tvorbě čtverce je vygenerován první bod, od nějž jsou následně zbylé body vypočteny.

\subsubsection{Random}
Body byly náhodně rozmístěny v rámci zobrazovacího okna.

\subsection{Konstrukce striktně konvexních obálek}
Striktně konvexní obálky jsou takové, že po sobě následující 3 body neleží na stejné přímce. V případě vygenerovaného čtverce o sto bodech bude striktně konvexní obálka obsahovat pouze 4 body, a to vrcholy. Pro tvorbu striktně konvexních obálek byla po vygenerování klasických obálek volána funkce na určení pozice bodu vůči linii. Pokud byl následující bod vyhodnocen tak, že leží na orientované úsečce $\overrightarrow{AB}$, bod B byl z množiny odebrán a následně byla testována další trojice bodů. Zároveň bylo nutné odebrat duplicitní body.


\subsection{Konstrukce Minimum Area Enclosing Box}
V rámci bonusových úloh bylo i sestrojení obdélníku s minimální plochou s využitím konvexní obálky. Tvorba takovýchto obdélníků se využívá v kartografii pro detekci tvaru a natočení budov. Případně by se mohla metoda používat v rámci generalizace, zde by však bylo zapotřebí ošetřit, aby výsledná data byla topologicky čistá. Při tvorbě minimálního obdélníku s využitím konvexní obálky se vychází z předpokladu, že nejméně jedna strana obdélníka je kolineární se stranou konvexní obálky. Složitost takovéto konstrukce má časovou náročnost O(n). 



\subsubsection{Implementace metody}
\begin{enumerate}
\item Inicializuj minimální plochu: $ A_{min} = \infty$
\item Inicializace místního souřadnicového systému a výpočet úhlu následujících bodů: $p=[0,0]; k=[0,\overrightarrow{PK}]$
\item Pro všechny body převod do místního systému: $points_i = p; points_{i+1} = k$
\item Nalezení Minimum Bounding Rectangle, který se dotýká konvexní obálky v extrémních bodech (min X, max X, min Y, max Y)
\item Porovnání s minimální plochou. $A < A_{min}$
\subitem Nahraď:$ A_{min}; Save (min X, max X, min Y, max Y); Save Angle $
\item Vytvoř obdélník z (min X, max X, min Y, max Y).
\item Nalezení největší osy obdélníku v místní soustavě.
\item Transformace do globálního systému.
\end{enumerate}



\section{Vstupní data}
Vstupními daty je množina bodů, která je vygenerovaná uživatelem. Uživatel může v tomto případě generovat body sám za pomoci kurzoru. Případně je v aplikaci implementována funkce pro automatické generování bodů. Uživatel má navíc možnost vlastní volby, jakým způsobem budou body generovány, zda budou body v kruhu, ve čtverci, elipse či v gridu. \\
\\
Při automatickém generování bodů je nutné nastavit i počet bodů, ze kterých bude daný útvar tvořen. Ne všechny tvary však lze vytvořit z vloženého množství bodů. V případě, že vstup nebude zcela odpovídat zvolenému útvaru, zobrazí se uživateli vyskakovací okno, které upozorní na nevhodný vstup a případnou modifikaci vloženého počtu. Tento případ nastane například při vložení tří bodů, ze kterých by uživatel chtěl vytvořit čtverec. 



\section{Výstupní data}
Výstupem této úlohy je grafická aplikace, ve které jsou vstupní data analyzována a následně je v závislosti na jejich vzájemně poloze vytvořena konvexní obálka. Konvexní obálku je možné vytvořit čtyřmi způsoby za pomoci nejznámějších algoritmů - Jarvis Scan, Quick Hull, Sweep Line a Graham Scan.\\
\\
Zároveň jsou výstupem grafy a tabulky s porovnáním jednotlivých dob výpočtů na různém množství dat. 

\clearpage
\section{Aplikace}
\begin{figure}[h!]
	\centering
	\includegraphics[width=10cm]{vstup.jpg}
	\caption{Zobrazené okno po spuštění aplikace}
\end{figure}

\begin{figure}[h!]
	\centering
	\includegraphics[width=10cm]{circle.jpg}
	\caption{Vygenerované body v kruhu}
\end{figure}

\begin{figure}[h!]
	\centering
	\includegraphics[width=10cm]{square.jpg}
	\caption{Vygenerované body ve čtverci}
\end{figure}

\begin{figure}[h!]
	\centering
	\includegraphics[width=10cm]{ellipse.jpg}
	\caption{Vygenerované body v elipse}
\end{figure}

\begin{figure}[h!]
	\centering
	\includegraphics[width=10cm]{grid.jpg}
	\caption{Vygenerované body v pravidelné mřížce}
\end{figure}

\begin{figure}[h!]
	\centering
	\includegraphics[width=10cm]{random.jpg}
	\caption{Náhodně vygenerované body}
\end{figure}

\begin{figure}[h!]
	\centering
	\includegraphics[width=10cm]{star_shaped.jpg}
	\caption{Vygenerované body ve tvaru Star Shaped}
\end{figure}

\begin{figure}[h!]
	\centering
	\includegraphics[width=10cm]{warning_grid.jpg}
	\caption{Varování na redukci vloženého počtu bodů při tvorbě gridu }
\end{figure}

\begin{figure}[h!]
	\centering
	\includegraphics[width=10cm]{warning_square.jpg}
	\caption{Varování při neplatném vstupu při tvorbě čtverce}
\end{figure}

\begin{figure}[h!]
	\centering
	\includegraphics[width=10cm]{warning_square2.jpg}
	\caption{Varování na redukci vloženého počtu bodů při tvorbě čtverce}
\end{figure}

\begin{figure}[h!]
	\centering
	\includegraphics[width=10cm]{hint.jpg}
	\caption{Nápověda, pokud si uživatel nebude jistý, jak v aplikaci postupovat}
\end{figure}

\begin{figure}[h!]
	\centering
	\includegraphics[width=10cm]{tooltip.jpg}
	\caption{Po přiložení kurzoru se zobrazí informace o objektu}
\end{figure}

\begin{figure}[h!]
	\centering
	\includegraphics[width=10cm]{close.jpg}
	\caption{Vyskakovací okno při ukončování aplikace}
\end{figure}

\begin{figure}[h!]
	\centering
	\includegraphics[width=10cm]{jarvis_ch.jpg}
	\caption{Tvorba konvexní obálky metodou Jarvis Scan}
\end{figure}

\begin{figure}[h!]
	\centering
	\includegraphics[width=10cm]{sweepline_ch.jpg}
	\caption{Tvorba konvexní obálky metodou Sweep Line}
\end{figure}

\begin{figure}[h!]
	\centering
	\includegraphics[width=10cm]{qhull_ch.jpg}
	\caption{Tvorba konvexní obálky metodou Quick Hull}
\end{figure}

\begin{figure}[h!]
	\centering
	\includegraphics[width=10cm]{grahamscan_ch.jpg}
	\caption{Tvorba konvexní obálky metodou Graham Scan}
\end{figure}

\begin{figure}[h!]
	\centering
	\includegraphics[width=10cm]{minBound.jpg}
	\caption{Tvorba minimálního ohraničujícího obdélníku}
\end{figure}

\clearpage

\section{Dokumentace}
\subsection{Třídy}
\subsubsection{Algorithms}
Třída Algorithms obsahuje několik metod. Metody jsou určeny pro výpočty použitých algoritmů.
\\

\textbf{double length2Points(QPoint q, QPoint p)}\\
Metoda, jejíž návratová hodnota je typu double, vrací velikost spojnice mezi dvěma body.
\\

\textbf{void polygonTransform(QPoint p, QPoint k, QPoint p1, QPoint k1, QPolygon \&pol)}\\
Metoda je přetížená pro  podobnostní transformaci QPolygon a QLine. Transformační klíč je dán dvěma body se souřadnicemi v obou souřadných soustavách. Body p a k jsou v souřadné soustavě, ze které pochází pol. Body p1 a k1 jsou v souřadné soustavě, do které chceme transformovat. Výsledek transformace se přeuloží do proměnné pol.
\\

\textbf{rotateByAngle(QPolygon \&points, double angle)}\\
Metoda je přetížená pro rotaci QPolygon a QLine. Prvky proměnné pol se pouze pootočí podle úhlu angle. Proměnná úhel je v radiánech. 
\\

\textbf{TPosition getPointLinePosition(QPoint \&q, QPoint \&a, QPoint \&b)}\\
Tato metoda slouží k určení pozice bodu q vůči linii tvořené body a, b. Výstupem metody je LEFT, RIGHT nebo ON.\\


\textbf{double get2LinesAngle(QPoint \&p1,QPoint \&p2,QPoint \&p3, QPoint \&p4)}\\
Pomocí této funkce je vypočítán úhel mezi dvěma přímkami, kde první přímka je tvořena body $ p_1 $ a $p_2$, druhá přímka je tvořena body $ p_3 $ a $p_4$.\\


\textbf{QPolygon CHJarvis (vector\textless QPoint \textgreater \&points)}\\
Metoda pro výpočet konvexní obálky nad vektorem bodů metodou Jarvis Scan. Metoda vrácí konvexní obálku s typem QPolygon.
\\

\textbf{QPolygon QHull (vector\textless QPoint \textgreater \&points)}\\
Metoda pro výpočet konvexní obálky nad vektorem bodů metodou Quick Hull. Metoda vrácí konvexní obálku s typem QPolygon.
\\

\textbf{void qh (int s, int e, vector\textless QPoint \textgreater \&p, QPolygon \&h)}\\
Pomocná metoda pro rekurzi v metodě Qhull. Na vstupu jsou indexy bodů s a e, které určují přímku, podle níž se určí, zda bod p patří do konvexní obálky H. Metoda nic nevrací, pouze ukládá body, které patří do konvexní obálky.
\\

\textbf{QPolygon GrahamScan (vector\textless QPoint \textgreater \&points)}\\
Metoda pro výpočet konvexní obálky nad vektorem bodů metodou Graham Scan. Metoda vrácí konvexní obálku s typem QPolygon.
\\

\textbf{QPolygon CHSweep (vector\textless QPoint \textgreater \&points)}\\
Metoda pro výpočet konvexní obálky nad vektorem bodů metodou Sweep Line. Metoda vrácí konvexní obálku s typem QPolygon.
\\

\textbf{void minimumAreaEnclosingBox (QPolygon \&ch, QPolygon \&rectangle, QLine \&direction)}\\
Metoda pro výpočet hlavních směrů budovy. Na vstupu je konvexní obálka ch a prázdné proměnné rectangle a direction. Do rectangle se uloží minimální ohraničující obdélník a do direction hlavní směr budovy.
\\

\textbf{QPolygon GrahamScanNew (vector\textless QPoint \textgreater \&points)}\\
Opravený algoritmus Graham Scan, který funguje i na větší množství bodů.
\\

\textbf{QPolygon deleteDuplicityCH (QPolygon ch)}\\
Metoda pro odstranění duplicitních bodů.
\\

\textbf{QPolygon exactlyCH (QPolygon ch)}\\
Metoda pro výpočet striktně konvexní obálky.
\\

\subsubsection{Draw}
Třída Draw obsahuje několik metod. Metody jsou určeny pro generování a vykreslování proměných.
\\

\textbf{void paintEvent(QPaintEvent *e)}\\
Metoda slouží k vykreslení vytvořených, generovaných bodů a zobrazení výsledků použitých algoritmů.
\\

\textbf{void mousePressEvent(QMouseEvent *e)}\\
Metoda uloží bod se souřadnicemi místa kliknutí v zobrazovacím okně.
\\

\textbf{void clearCanvas()}\\
Metoda slouží k vymazání proměnných a k překreslení
\\

\textbf{std::vector\textless QPoint \textgreater generateGrid(int n)}\\
Metoda generuje pravidelnou mřížku. Na vstupu je počet generovaných bodů. Návratová hodnota je vektor bodů.\\

\textbf{std::vector\textless QPoint \textgreater generateRandomPoints(int n)}\\
Metoda generuje náhodné body. Na vstupu je počet generovaných bodů. Návratová hodnota je vektor bodů.\\

\textbf{std::vector\textless QPoint \textgreater generateStarShape(int n)}\\
Metoda generuje body do tvaru hvězdy. Na vstupu je počet generovaných bodů. Návratová hodnota je vektor bodů.\\

\textbf{std::vector\textless QPoint \textgreater generateSquare(int n)}\\
Metoda generuje body do tvaru čtverce. Na vstupu je počet generovaných bodů, změna v počtu bodů se projení s násobkem čtyř. Návratová hodnota je vektor bodů.\\

\textbf{std::vector\textless QPoint \textgreater generateEclipse(int n)}\\
Metoda generuje body do tvaru elipsy. Na vstupu je počet generovaných bodů. Návratová hodnota je vektor bodů.\\

\textbf{std::vector\textless QPoint \textgreater generateCircle(int n)}\\
Metoda generuje body do tvaru kruhu. Na vstupu je počet generovaných bodů. Návratová hodnota je vektor bodů.\\

\textbf{void setCH(QPolygon ch\_)}\\
Metoda slouží pro převod konvexní obálky do vykreslovacího okna.\\

\textbf{setRectangle(QPolygon rectangle\_)}\\
Metoda slouží pro převod minimální ohraničujícího obdélníku do vykreslovacího okna.\\

\textbf{setDirection(QLine direction\_)}\\
Metoda slouží pro převod hlavního směru budovy do vykreslovacího okna.\\

\textbf{setPoints(std::vector\textless QPoint \textgreater points\_)}\\
Metoda slouží pro převod vektoru bodů do vykreslovacího okna.\\

\textbf{std::vector\textless QPoint \textgreater getPoints()}\\
Metoda slouží pro převod vektoru bodů z vykreslovacího okna.\\

\textbf{QPolygon getConvexHull()}\\
Metoda slouží pro převod konvexní obálky z vykreslovacího okna.

\subsubsection{SortByXAsc}
Třída SortByXAsc slouží k porovnání souřadnic v ose x.\\


\textbf{bool operator()(QPoint \&p1, QPoint \&p2)}\\
Přetížený operátor () vrátí bod s větší souřadnicí x z dvojice bodů.\\


\subsubsection{sortByYAsc}
Třída sortByYAsc slouží k porovnání souřadnic v ose y.
\\

\textbf{bool operator()(QPoint \&p1, QPoint \&p2)}\\
Přetížený operátor () vrátí bod s větší souřadnicí y z dvojice bodů.
\\

\subsubsection{Widget}


\textbf{void on\_pushButton\_clicked()}\\
Při stisknutí tlačítka Create CH se volají metody třídy Algorithms dle metody vybrané v comboboxu.
\\

\textbf{void on\_generate\_clicked()}\\
Při stisknutí tlačítka Generate se volají metody třídy Draw dle metody vybrané v comboboxu.
\\

\textbf{void on\_pushButton\_2\_clicked()}\\
Při stisknutí tlačítka Clear se zavolá metoda třídy Draw clearCanvas.
\\

\textbf{void on\_minimumAreaEnclosingBox\_clicked()}\\
Při stisknutí tlačítka MinEndBox se nad konvexní obálkou zavolá metoda třídy Algorithms minimumAreaEnclosingBox.
\\

\textbf{void on\_pushButton\_3\_clicked()}\\
Při stisknutí tlačítka Close je uživatel vyzván, zda chce skutečně aplikaci ukončit.
\\

\textbf{void on\_pushButton\_4\_clicked()}\\
Při stisknutí tlačítka Hint je uživateli zobrazena nápověda, jakým způsobem postupovat, neboť při stisknutí tlačítka pro tvorbu konvexní obálky nad prázdnou množinou je aplikace ukončena.
\\


\section{Testování}
Po vytvoření funkční aplikace byly jednotlivé algoritmy otestovány na různém množství dat a na různých typech množin. Testování bylo prováděno nad výpočtem striktně konvexních obálek, které jsou výstupem aplikace. Samotné testování bylo prováděno v režimu Release. Testované typy množin byly voleny kruh, grid a náhodné rozmístění. Jednotlivé doby byly uloženy do tabulek a následně zobrazeny v grafu. Přehled veškerých dat je v přiloženém excell souboru včetně rozptylu naměřených dob. V následujících podkapitolách jsou k dispozici průměrné doby jednotlivých algoritmů.

\clearpage
\subsection{Jarvis Scan}


\begin{table}[h!]
	\centering
	\begin{tabular}{|r|r|r|r|}
	\hline
	 \textbf{Počet bodů} 	& \textbf{Circle [ms]} & \textbf{Grid [ms]}  & \textbf{Random[ms]} \\ \hline
	 1000 & 14.2 & 0.1 & 1.2   \\ \hline
	10000 & 132.5 & 4.7 & 33.8   \\ \hline
	100000 & 1791.8 & 48.8 & 46.2  \\ \hline
	225000 & 3135.1 & 102.9 & 106.2   \\ \hline
	350000 & 5294.7 & 160.1 & 165.8 \\ \hline
	475000 & 7772.3 & 218.4 & 226.1  \\ \hline
	600000 & 8327.2 & 279.5 & 286.1   \\ \hline
	725000 & 9379.4 & 335.4 & 345.2  \\ \hline
	850000 & 11695.6 & 394.3 & 404.9  \\ \hline
	1000000 & 13820.4 & 526.9 & 475.9  \\ \hline
	
	\end{tabular}
		\caption{Sweep Line}
\end{table}



\subsection{Sweep Line}



\begin{table}[h!]
	\centering
	\begin{tabular}{|r|r|r|r|}
	\hline
	 \textbf{Počet bodů} 	& \textbf{Circle [ms]} & \textbf{Grid [ms]}  & \textbf{Random[ms]} \\ \hline
	 1000 & 0 & 0 & 0   \\ \hline
	10000 & 0.7 & 0.7 & 0.8   \\ \hline
	100000 & 2.2 & 4.7 & 9.3  \\ \hline
	225000 & 4.8 & 10.8 & 22.8   \\ \hline
	350000 & 8.7 & 16.8 & 34.2  \\ \hline
	475000 & 10.9 & 22.7 & 47.9  \\ \hline
	600000 & 15.7 & 29.4 & 59.5   \\ \hline
	725000 & 17.7 & 35.4 & 71.3  \\ \hline
	850000 & 20.2 & 41.4 & 81.9  \\ \hline
	1000000 & 24.1 & 47.9 & 97.2  \\ \hline
	
	\end{tabular}
		\caption{Sweep Line}
\end{table}

\clearpage


\subsection{Quick Hull}


\begin{table}[h!]
	\centering
	\begin{tabular}{|r|r|r|r|}
	\hline
	 \textbf{Počet bodů} 	& \textbf{Circle [ms]} & \textbf{Grid [ms]}  & \textbf{Random[ms]} \\ \hline
	 1000 & 0 & 0.1 & 0   \\ \hline
	10000 & 2.4 & 0.8 & 1.3   \\ \hline
	100000 & 16.4 & 4.7 & 11.1  \\ \hline
	225000 & 56.4 & 10.2 & 24.5   \\ \hline
	350000 & 86.7 & 15.6 & 38.2  \\ \hline
	475000 & 127.1 & 20.5 & 52.4  \\ \hline
	600000 & 145.5 & 27.5 & 66.4   \\ \hline
	725000 & 209.6 & 32.8 & 80  \\ \hline
	850000 & 236.2 & 37.6 & 95.4  \\ \hline
	1000000 & 278.1 & 43.8 & 110.4  \\ \hline
	
	\end{tabular}
		\caption{Quick Hull}
\end{table}




\subsection{Graham Scan}


\begin{table}[h!]
	\centering
	\begin{tabular}{|r|r|r|r|}
	\hline
	 \textbf{Počet bodů} 	& \textbf{Circle [ms]} & \textbf{Grid [ms]}  & \textbf{Random[ms]} \\ \hline
	 1000 & 0.3 & 0.4 & 0   \\ \hline
	10000 & 1.8 & 3.2 & 3.3   \\ \hline
	100000 & 18.6 & 35.5 & 34.3  \\ \hline
	225000 & 42.1 & 76.1 & 77.8   \\ \hline
	350000 & 68.1 & 121.5 & 122.7  \\ \hline
	475000 & 91.2 & 167.6 & 164 \\ \hline
	600000 & 120.9 & 220 & 212   \\ \hline
	725000 & 146.1 & 270.8 & 252.7  \\ \hline
	850000 & 167.6 & 313.2 & 292.8  \\ \hline
	1000000 & 197.3 & 444.4 & 340.4  \\ \hline
	
	\end{tabular}
		\caption{Graham Scan}
\end{table}

\clearpage
\subsection{Grafy algoritmu Jarvis Scan}
\begin{figure}[h!]
	\centering
	\includegraphics[width=15cm]{grafy/js_circle.png}
\end{figure}

\begin{figure}[h!]
	\centering
	\includegraphics[width=15cm]{grafy/js_grid.png}
\end{figure}

\clearpage
\begin{figure}[h!]
	\centering
	\includegraphics[width=15cm]{grafy/js_random.png}
\end{figure}

\begin{figure}[h!]
	\centering
	\includegraphics[width=15cm]{grafy/js.png}
\end{figure}


\clearpage
\subsection{Grafy algoritmu Sweep Line}
\begin{figure}[h!]
	\centering
	\includegraphics[width=15cm]{grafy/sl_circle.png}
\end{figure}

\begin{figure}[h!]
	\centering
	\includegraphics[width=15cm]{grafy/sl_grid.png}
\end{figure}

\clearpage
\begin{figure}[h!]
	\centering
	\includegraphics[width=15cm]{grafy/sl_random.png}
\end{figure}

\begin{figure}[h!]
	\centering
	\includegraphics[width=15cm]{grafy/sl.png}
\end{figure}

\clearpage
\subsection{Grafy algoritmu Quick Hull}
\begin{figure}[h!]
	\centering
	\includegraphics[width=15cm]{grafy/qh_circle.png}
\end{figure}

\begin{figure}[h!]
	\centering
	\includegraphics[width=15cm]{grafy/qh_grid.png}
\end{figure}

\clearpage
\begin{figure}[h!]
	\centering
	\includegraphics[width=15cm]{grafy/qh_random.png}
\end{figure}

\begin{figure}[h!]
	\centering
	\includegraphics[width=15cm]{grafy/qh.png}
\end{figure}

\clearpage
\subsection{Grafy algoritmu Graham Scan}
\begin{figure}[h!]
	\centering
	\includegraphics[width=15cm]{grafy/gs_circle.png}
\end{figure}

\begin{figure}[h!]
	\centering
	\includegraphics[width=15cm]{grafy/gs_grid.png}
\end{figure}

\clearpage
\begin{figure}[h!]
	\centering
	\includegraphics[width=15cm]{grafy/gs_random.png}
\end{figure}

\begin{figure}[h!]
	\centering
	\includegraphics[width=15cm]{grafy/gs.png}
\end{figure}

\clearpage
\subsection{Porovnání jednotlivých metod na použitých množinách}
\begin{figure}[h!]
	\centering
	\includegraphics[width=15cm]{grafy/kruh.png}
\end{figure}

\begin{figure}[h!]
	\centering
	\includegraphics[width=15cm]{grafy/grid.png}
\end{figure}

\clearpage
\begin{figure}[h!]
	\centering
	\includegraphics[width=15cm]{grafy/random.png}
\end{figure}





\clearpage
\section{Závěr}
V rámci úlohy byla vytvořena aplikace, která je schopna na vygenerovaných bodech zkonstruovat konvexní obálku. Zároveň je v rámci aplikace možné náhodně vygenerovat téměř libovolné množství bodů do určitého tvaru. Na výběr je poměrně široké spektrum běžných tvarů, což jistě uživatel ocení.\\
\\
Součástí úlohy bylo i testování doby výpočtu jednotlivých konvexních obálek. Testování bylo prováděno nad striktně konvexními obálkami, samotný výpočet je tedy zatížen ještě o vyhodnocení, zda následující tři body neleží na stejné přímce. Výsledné doby dopadly dle očekávání. Algoritmus Jarvis Scan měl znatelně nejhorší čas. Naopak metoda Graham Scan se jeví jako nejlepší řešení.\\
\\
U metod Sweep Line a Jarvis Scan byla objevena chyba v kódu. Chyba způsobovala to, že výpočet algoritmů nebyl efektivní. Z toho důvodu bylo nutné kód opravit a provést znovu testování. Testování bylo následně provedeno na jiném PC, byly tedy otestovány všechny algoritmy, aby bylo možné výsledky porovnávat. Po opravě kódu již fungovala tvorba konvexní obálky i pro množiny bodů, u kterých funkce v předešlém kódu padala. \\
\\
Zároveň byl zkvalitněn výstup grafů. Původně byly grafy ukládány jako výstřižky, nyní jsou grafy z excellu exportovány, aby byla zachována kvalita. Práce v LaTeXu byla také zkvalitněna, tabulky byly vypracovány tak, aby byly čitelné a obsahovaly názvy sloupců a řádků. Součástí technické zprávy je nyní i excell s přehledem všech měření, včetně rozptylů.

\section{Náměty na vylepšení}
\subsection{Graham Scan}
Pro algoritmus Graham Scan byl nejprve napsán kod, který byl velmi náročný a aplikace nebyla funkční na větším množství dat. Z toho důvodu byla napsána nová metoda, která je již funkční a při testování dosahovala nejlepších možných výsledků. Z toho důvodu je v kódu pojmenována jako GrahamScanNew. Původní funkce byla v kódu ponechána pro případnou doeditaci a její zprovoznění.

\subsection{Generování množin bodů}
Při tvorbě generování gridu napadla autory myšlenka, že by se body generovaly v některém ze známých kartografických zobrazení. Z bakalářského studia znají velké množství zobrazovacích rovnic, pomocí nichž by se body zobrazovaly. Zajímavé rozmístění by měla zejména kuželová zobrazení. Při zobrazení bodů v rámci některého kartografického zobrazení by výsledná konvexní obálka představovala celý svět, případně polokouli, záleželo by na typu zvoleného zobrazení.

\subsection{Testování algoritmů}
V rámci této úlohy bylo provedeno několik testování nad odlišným množstvím bodů a na jejich odlišném zobrazení. Časově bylo testování poměrně náročné, za úvahu jistě stojí popřemýšlení, zda by nebylo výhodnější vytvořit takovou funkci, která by pro předem definované tvary a množství bodů sama provedla testování. Zároveň obsahuje platforma Qt i knihovnu grafů, časová úspora by tedy byla velmi výrazná.


\subsection{Minimum Area Enclosing Box}
Při výpočtu minimálního ohraničujícího obdélníku autorům z neznámé příčiny nefungovala transformace, ač se to snažili vyřešit. Po konzultaci byla část pro transformaci vypůjčena od kolegy, kterému fungovala bez problémů. Vypůjčená část je v kódu vyznačena. 





\clearpage
\section{Reference}

\begin{enumerate}
\item  MARTÍNEK, Petr. Konvexní obálka rozsáhlé množiny bodů v E\^d [online][cit. 31.10.2018]. \\
Dostupné z: http://graphics.zcu.cz/files/86\_BP\_2010\_Martinek\_Petr.pdf  \\

\item Convex Hulls: Explained. [online][cit. 31.10.2018]\\
Dostupné z: https://medium.com/@harshitsikchi/convex-hulls-explained-baab662c4e94\\

\item Convex-hull mass estimates of the dodo (Raphus cucullatus). [online][cit. 31.10.2018]\\
Dostupné z: https://www.semanticscholar.org/paper/Convex-hull-mass-estimates-of-the-dodo-(Raphus-of-a-Brassey-O\'Mahoney/12e07d3b712561cad16501ac8096120e14901eb8

\item GeeksforGeeks: Convex Hull - Set 1 (Jarvis's Algorithm or Wrapping. [online][cit. 5.11.2018]\\
Dostupné z: https://www.geeksforgeeks.org/convex-hull-set-1-jarviss-algorithm-or-wrapping/

\item  BAYER, Tomáš. Geometrické vyhledávání [online][cit. 5.11.2018]. \\
Dostupné z: https://web.natur.cuni.cz/~bayertom/images/courses/Adk/adk4.pdf  \\

\item GeeksforGeeks: Convex Hull - Set 2 (Graham Scan). [online][cit. 12.11.2018]\\
Dostupné z: https://www.geeksforgeeks.org/convex-hull-set-2-graham-scan/



\end{enumerate}
\end{document}



 