\documentclass[a4paper, 12pt]{article}
\usepackage[total={17cm,25cm}, top=2.5cm, left=2.5cm, right=2.5cm,  includefoot]{geometry}
\usepackage[utf8]{inputenc}
\usepackage{array}
\usepackage{multirow}
\usepackage{hhline}
\usepackage{gensymb}
\usepackage{graphicx}
\graphicspath{ {} }
\usepackage[czech]{babel}
\usepackage{enumitem}
\usepackage{pdfpages}
\usepackage{amsmath}
\usepackage{verbatim}
\usepackage{listings}
\usepackage{hyperref}
\usepackage{amssymb}


\pagestyle{empty} % vypne číslování stránek




\usepackage[OT2,OT1]{fontenc}
\newcommand\cyr
{
\renewcommand\rmdefault{wncyr}
\renewcommand\sfdefault{wncyss}
\renewcommand\encodingdefault{OT2}
\normalfont
\selectfont
}
\DeclareTextFontCommand{\textcyr}{\cyr}
\def\cprime{\char"7E }
\def\cdprime{\char"7F }
\def\eoborotnoye{\char’013}
\def\Eoborotnoye{\char’003}


\begin{document}



\begin{titlepage}
\begin{center}
\noindent
\Large \textbf{České vysoké učení technické v Praze }\\ Fakulta stavební
\vspace{5cm}

\huge

%vložení loga cvut
\begin{figure}[h!]
	\centering
	\includegraphics[width=7cm]{pictures/logo.png}
\end{figure}

\vspace{0.5cm}

Algoritmy v digitální kartografii \\

\vspace{3cm}

\Huge  
Množinové operace s polygony\\

\vspace{2cm}

\Large
Bc. Petra Pasovská \\
Bc. David Zahradník \\

\end{center}

\end{titlepage}




\pagestyle{plain}     % zapne obyčejné číslování
\setcounter{page}{1}  % nastaví čítač stránek znovu od jedné

\tableofcontents
\newpage

\section{Zadání}
Níže uvedené zadání je kopie ze stránek předmětu. 

\begin{figure}[h!]
	\centering
	\includegraphics[width=17cm]{zadani.pdf}
\end{figure}




\section{Popis a rozbor problému}
Hlavním cílem této úlohy je tvorba aplikace, která je schopná na vygenerovaných polygonech provádět základní množinové operace. V rámci této úlohy je tedy možné vypočítat průnik, sjednocení a rozdíl dvou polygonů.\\
\\
Obecně lze pro určení vztahů použít tzv. Booleovské operátory. Jsou pojmenovány po Georgovi Booleovi, který je použil ve své knize z roku 1854. Základní Booleovské operátory jsou AND (průnik - logický součin), OR (sjednocení - logický součet) a NOT (negace). V Booleovské logice mohou nabývat datové typy bool dvou hodnot, a to 1 (TRUE) či 0 (FALSE). \\
\\
Množinové operace mají v kartografii velké využití, zejména v prostředí GIS softwarů. Ve většině těchto programů jsou již naimplementovány základní funkce včetně jejich rozšíření. Mezi jedny z nejčastějších rozšířených funkcí patří funkce buffer, která vytváří obalovou zónu kolem vybraných dat. Z praktického hlediska je poté buffer využíván například pro hledání objektů v určité vzdálenosti apod.

\begin{figure}[h!]
	\centering
	\includegraphics[width=7cm]{pictures/operace.png}
	\caption{Přehled základních množinových operací [zdroj: 1]}
\end{figure}

\subsection{Problematické situace}
Problematické situace vznikají při:

\begin{enumerate}
\item Společný vrchol
\item Společné vrcholy
\item Vrchol leží na hraně druhého polygonu
\item Společná část elementu
\item Společný element
\item Element leží v elementu druhého polygonu
\end{enumerate}

Pokud by měly polygony společný vrchol/vrcholy/vrchol na hraně, tak výsledkem operace UNION by byla prázdná množina. Polygon, který by měl vrchol větší než stupeň dva, by byl nekorektní.\\

Při operaci INTERSECT výše zmíněných situací, by výsledek byl bud 0D entita (bod) nebo 1D entita (úsečka).

Výsledek operace DIFFERENCE výše zmíněných situací by měl být jeden ze vstupních polygonů.

\begin{figure}[h!]
	\centering
	\includegraphics[width=7cm]{pictures/problem.png}
	\caption{Problémové situace}
\end{figure}

\section{Popis použitých algoritmů}
V této úloze byl vytvořen nový datový typ - QPointFB. Pro tvorbu této aplikace bylo použito několik dílčích algoritmů. Obecně lze postup rozdělit na tyto fáze:

\begin{enumerate}
\item Výpočet průsečíků A, B + setřídění
\item Update A, B
\item Ohodnocení vrcholů A, B, dle pozice vůči B, A
\item Výběr vrcholů dle ohodnocení
\item Vytvoření fragmentů
\item Vytvoření oblastí z fragmentů
\item Nastavení orientace polygonů
\end{enumerate}

\subsection{Výpočet průsečíků A, B + setřídění}
V rámci výpočtu průsečíků se prochází jednotlivé body polygonů a porovnává se jejich vzájemný vztah. K určení vztahu byla použita již existující funkce s názvem Get2LinesPosition. Pokud byly linie vyhodnoceny tak, že se protínají, byl průsečík uložen do nově vytvořené proměnné o datovém typu mapa. V rámci proměnné mapa se ukládá výsledek a tzv. klíč, který k dané hodnotě odkazuje. Klíč je označen jako $\alpha$. Časová náročnost tohoto výpočtu je rovna O(m,n).

\subsubsection{Implementace metody}
\begin{enumerate}
\item Pro všechna i: $for (i = 0; i < n; i++)$
\item Vytvoření mapy: $ M = map<double, QPointFB>$
\item Pro všechna j: $for (j = 0; j < m; j++)$
\subitem Pokud existuje průsečík: $if (b_{ij} = (p_i, p_{(i+1)\%n}) \cap (q_j, q_{(j+1)\%m})  \neq \emptyset  )$
\subitem Přidání do mapy: $M[\alpha_i] \leftarrow b_{ij}$
\subitem Zpracování prvního průsečíku: $ProcessIntersection (b_{ij}, \beta, B, j)$
\item Při nalezení průsečíků: $if (\| M \| > 0) $
\subitem Procházení všech průsečíků: $for \forall m \in M$
\subitem Zpracování aktuálního průsečíku: $ProcessIntersection(b, \alpha, A, i)$

\end{enumerate}


\subsection{Update A, B}
S nalezením každého průsečíku je nutné updatovat seznam bodů obou polygonů. K tomu slouží funkce ProcessIntersection. Při nalezení správné pozice je nový bod vložen za pomoci defaultní funkce insert, do níž vstupuje nejprve pozice umístění a následně hodnota, která se vkládá.

\subsubsection{Implementace metody}
Do metody ProcessIntersection vstupuje bod, který je průsečíkem, koeficient přímky alfa nebo beta, polygon a index, prvního bodu přímky, na kterém leží průsečík.


\begin{enumerate}
\item Pokud koeficient t se blíží nule. $ if( abs(t) < eps )$
\subitem Označ počáteční vrchol přímky jako průsečík. $ poly[i] is INTERSECT $
\item Pokud koeficient t je blíží k jedné. $if( abs(abs(t)-1) < eps ) $
\subitem Označ koncový vrchol přímky jako průsečík. $ poly[i+1] is INTERSECT $
\item Je-li koeficient mezi nulou a jednou. $ else $
\subitem Vlož průsečík na následují pozici po indexu. $ insert INTERSECT to poly[i+1] $
\end{enumerate}

\subsection{Ohodnocení vrcholů A, B, dle pozice vůči B, A}
Pro možnost ohodnocení vrcholů vůči jednotlivým polygonům byl vytvořen nový datový typ. Tento datový typ nabývá hodnot podle toho, zda se bod nachází uvnitř polygonu, na hranici či mimo polygon. 

\subsubsection{Implementace metody}
Do metody setPositions vstupují oba polygony.

\begin{enumerate}
\item Pro všechny body polygonu A. $ for( all points in polygon A ) $
\subitem Vypočti střed hrany polygonu. $ x/y = (poly[i] + poly[i+1])/2 $
\subitem Zjisti pozici středu oproti polygonu B . $ getPossitionWinding() $
\subitem Nastav pozici počátečnímu bodu hrany . $ set loc = IN/OUT/BOUNDARY $
\item Pro všechny body polygonu B. $ for( all points in polygon B ) $
\subitem Vypočti střed hrany polygonu. $ x/y = (poly[i] + poly[i+1])/2 $
\subitem Zjisti pozici středu oproti polygonu A . $ getPossitionWinding() $
\subitem Nastav pozici počátečnímu bodu hrany . $ set loc = IN/OUT/BOUNDARY $
\end{enumerate}


\subsection{Vytvoření fragmentů}
Vrcholy se stejným ohodnocením byly přidány do polylinie, respektive do fragmentu. Body jsou uloženy včetně pozice, na níž se nachází. Každý fragment začíná průsečíkem a končí prvním bodem s jiným ohodnocením. Pro diferenci má fragment opačné pořadí vrcholů - po směru hodinových ručiček. 

\subsubsection{Implementace metody}
Do této metody vstupuje polygon P o n velikosti, ohodnocením vrcholů g, změnou orientace s a seznamem fragmentů F. 

\begin{enumerate}
\item Dokud P[i] není průsečík s orientací g: $g(P[i]) \neq g \vee P[i] \neq inters )$
\subitem $i \leftarrow i + 1 $
\item Žádný bod s touto orientací neexistuje: $if (i \equiv n) return$
\item Uložení startovního indexu prvního průsečíku: $i_s \leftarrow i$
\subitem Vytvoření prázdného fragmentu: $f = \emptyset $
\subitem Při nalezení fragmentu: 
\subitem Swapování prvků je-li potřeba: $f.reverse()$
\subitem Přidání fragmentu do mapy s klíčem počátečního bodu: $F[f[0]] \rightarrow f$
\subitem Přejdi k dalšímu bodu přes index: $i \leftarrow (i+1)\% m$
\item Opakování dokud se nevrátí zpět k počátečnímu průsečíku: $ while (i \neq i_s)$
\end{enumerate}

Následně byla vytvořena ještě jedna funkce pro tvorbu fragmentů, do níž vstupuje index startovního bodu, polygon P, orientace g, index vrcholů i a již vytvořený fragment f.

\begin{enumerate}
\item Bod není průsečíkem s orientací g: $if g(P[i]) \neq g \vee P[i] \neq inters \rightarrow return FALSE$
\item Nekonečný cyklus: $for (;;)$
\subitem Přidání bodu do fragmentu: $f \leftarrow P[i]$
\subitem Následující bod ze seznamu: $i \leftarrow (i+1)\%n$
\subitem Při navrácení ke startovnímu bodu: $if (i \equiv i_s) \rightarrow return FALSE$
\subitem Při nalezení prvního bodu s rozdílnou orientací: $if (g(P[i]) \neq g)$
\subitem Přidání bodu do seznamu a úspěšné ukončení: $f \leftarrow P[i] \rightarrow return TRUE$
\end{enumerate}


\subsection{Vytvoření oblastí z fragmentů}
Následně je nutné projít všechny fragmenty a sestavit z nich oblasti. Vstupem do funkce jsou vzniklé fragmenty F a výstupem seznam polygonů C.

\subsubsection{Implementace metody}
\begin{enumerate}
\item Pro všechna f: $ for \forall f \in F$
\item Vytvoření prázdného polygonu: $P \leftarrow \emptyset$
\item Nalezení startovního bodu fragmentu: $s \leftarrow f.first$
\item Při nezpracování fragmentu: $if (!f.second.first)$
\subitem Přidání polygonu do seznamu: $C \leftarrow P$
\end{enumerate}

Z oblastí jsou následně vytvářeny polygony funkcí createPolygonFromFragments. n značí následující bod (next), s startovní bod (start). 

\begin{enumerate}
\item Inicializace následujícího bodu: $QPointFB n \leftarrow s$
\item Nekonečný cyklus k procházení všech fragmentů: $for(;;)$
\item Nalezení navazujícího fragmentu: $f \leftarrow F.find(n)$
\item Při neexistenci fragmentu s takovýmto počátečním bodem: $if(f \equiv F.end) \rightarrow return FALSE$
\item Fragment označen za zpracovaný: $f.second.first \leftarrow TRUE$
\item Následující bod: $n \leftarrow f.second.second.back()$
\item Přidání bez počátečního bodu: $P \leftarrow f.second.second - \{ f.second.second[0]\} $
\item Při vrácení se na začátek: $if (n \equiv s) \rightarrow return TRUE$
\end{enumerate}

\subsection{Nastavení orientace polygonů}
Pro správný výpočet Boolovských operací je potřeba, aby polygony měli CCW orientaci.

\subsubsection{Implementace metody}
\begin{enumerate}
\item Zjisti orientaci polygonu pomocí L´Huillierových vzorců.
\subsubitem Spočti výměru.
\item Pokud je výměra záporná prohoď orientaci polygonu.

\end{enumerate}


\subsection{Výsledný algoritmus}
Po vytvoření zmíněných dílčích algoritmů jsou funkce postupně volány.

\begin{enumerate}
\item Nastavení správné orientace obou polygonů.
\item Výpočet průsečíků A, B: $ComputeIntersections(A, B)$
\item Určení polohy vrcholů vůči oblastem: $setPositions(A, B) $
\item Tvorba mapy fragmentů: $map F$
\item Určení pozice: $pos = (oper \equiv Intersection \lor oper \equiv DifBA?Inner:Outer ) $
\item Swapnutí: $swap = (oper \equiv DifAB) : TRUE : FALSE$
\item Tvorba fragmentů: $createFragments(A, pos, swap, F)$
\item Propojení fragmentů: $mergeFragments(A, B, C)$
\end{enumerate}





\section{Vstupní data}
Vstupní data musí být seřazená, tedy každý polygon v souboru musí začínat bodem jedna a končit n-tým bodem.  Ve vstupních datech je polygon A dán číslem 1 a polygon B číslem jiným než 1. Jednotlivé body se sekvenčně ukládají do proměnné QPointFB a následně polygonu A/B.\\
\\
Struktura vstupních dat:
[číslo polygonu, souřadnice X, souřadnice Y]


\section{Výstupní data}
Výstupem aplikace je grafické znázornění jednotlivých booleovských metod na načtených polygonech, případně ručně vložených polygonech.


\clearpage
\section{Aplikace}

\begin{figure}[h!]
\centering
\includegraphics[width=10cm]{pictures/polygons.jpg}
\caption{Vzhled aplikace a ruční vložení obou polygonů}
\end{figure}

\begin{figure}[h!]
\centering
\includegraphics[width=10cm]{pictures/import.jpg}
\caption{Ukázka naimportovaných polygonů}
\end{figure}

\begin{figure}[h!]
\centering
\includegraphics[width=10cm]{pictures/union.jpg}
\caption{Výsledek operace sjednocení (UNION)}
\end{figure}

\begin{figure}[h!]
\centering
\includegraphics[width=10cm]{pictures/intersection.jpg}
\caption{Výsledek operace průnik (INTERSECTION}
\end{figure}

\begin{figure}[h!]
\centering
\includegraphics[width=10cm]{pictures/diffAB.jpg}
\caption{Výsledek operace rozdílu A a B (DIFFERENCE A - B)}
\end{figure}

\begin{figure}[h!]
\centering
\includegraphics[width=10cm]{pictures/diffBA.jpg}
\caption{Výsledek operace rozdílu B a A (DIFFERENCE B - A)}
\end{figure}

\begin{figure}[h!]
\centering
\includegraphics[width=10cm]{pictures/1D.jpg}
\caption{Ukázka aplikace v případě, že výsledek je linie (1D entita)}
\end{figure}

\begin{figure}[h!]
\centering
\includegraphics[width=10cm]{pictures/0D.jpg}
\caption{Ukázka aplikace v případě, že výsledek je bod (0D entita)}
\end{figure}

\clearpage



\section{Dokumentace}
\subsection{Třídy}
\subsubsection{Algorithms}
Třída Algorithms obsahuje několik metod. Metody jsou určeny pro výpočty použitých algoritmů.
\\

\textbf{TPointPolygon getPositionWinding(QPointFB q, std::vector\textless QPointFB\textgreater pol)}\\
Metoda, která vrátí vztah polohy bodu q a polygonu pol. Návratové hodnoty jsou INSIDE, OUTSIDE, ON .\\

\textbf{TPointLinePosition getPointLinePosition(QPointFB \&q, QPointFB \&a, QPointFB \&b)}\\
Metoda, která vrátí vztah polohy bodu q a přímky tvořenou body a a b. Návratové hodnoty jsou LEFT, RIGHT, COL (na hraně).\\

\textbf{double get2LinesAngle(QPointFB \&p1,QPointFB \&p2,QPointFB \&p3, QPointFB \&p4)}\\
Tato metoda slouží k vypočetní hodnoty úhlu mezi dvěma přímkami \textless p1,p2\textgreater a \textless p3,p4\textgreater .\\

\textbf{T2LinesPosition get2LinesPosition(QPointFB \&p1,QPointFB \&p2,QPointFB \&p3, QPointFB \&p4, QPointFB \&intersection)}\\
Metoda, která vrátí vztah dvou přímek a do proměné intersection, pokud existuje, spočte průsečík a přiřadí jemu hodnoty do typu QPointFB. Návratové hodnoty jsou PARALLEL, COLINEAR, INTERSECTING, NONINTERSECTING .\\


\textbf{void computePolygonIntersections(std::vector\textless QPointFB\textgreater \&p1, \\
std::vector\textless QPointFB\textgreater \&p2)}\\
Metoda spočte průsečíky polygonů p1 a p2 a vloží je do oněh polygonů (spustí se metoda processIntersection).\\

\textbf{void processIntersection(QPointFB \&b, double t, std::vector\textless QPointFB\textgreater \&poly, int \&i)}\\
Metoda vloží bod b do polygonu poly na pozici i+1 , pokud je na hraně daného polygonu a není vrcholem.\\

\textbf{void setPositions (std::vector\textless QPointFB\textgreater \&pol1,std::vector\textless QPointFB\textgreater \&pol2)}\\
Metoda nastaví bodu polygonu pol1 třídy QPointFB hodnotu pos, podle vztahu s polygonem pol2 a opačně.\\

\textbf{void createFragments(std::vector\textless QPointFB\textgreater \&pol, TPointPolygon posit, bool rev, std::map\textless QPointFB,std::vector\textless QPointFB\textgreater \textgreater  \&F)}\\
Metoda vytvoří fragmenty se stejnou hodnotou posit a uloží je do hasovací tabulky F.\\

\textbf{void mergeFragments(std::map\textless QPointFB, std::vector\textless QPointFB\textgreater \textgreater \&Fa,\\std::map\textless QPointFB, std::vector\textless QPointFB\textgreater \textgreater \&Fb, std::vector\textless std::vector\\ \textless QPointFB\textgreater \textgreater \&C)}\\
Metoda sjednotí fragmenty z polygonu A a polygonu C a uloží je do vektoru polygonů C.\\

\textbf{double getPolygonOrientation(std::vector\textless QPointFB\textgreater \&pol)}\\
Metoda vrátí plochu polygonu pol. Pokud je výměra záporná polygon na CCV orientaci.\\

\textbf{std::vector\textless std::vector\textless QPointFB\textgreater \textgreater BooleanOper(std::vector\textless QPointFB\textgreater \&A, std::vector\textless QPointFB\textgreater \&B, TBooleanOperation oper)}\\
Metoda nad polygony A a B provede metodu oper: INTERSECTION, UNION, DIFFAB, DIFFBA.\\


\subsubsection{Draw}
Třída Draw obsahuje několik metod. Metody jsou určeny pro generování a vykreslování proměných.
\\

\textbf{void paintEvent(QPaintEvent *e)}\\
Metoda pro kreslení do vykreslovacího okna.\\

\textbf{void drawPol(std::vector\textless QPointFB> \&pol, QPainter \&painter)}\\
Metoda pro vykreslení polygonu.\\

\textbf{void mousePressEvent(QMouseEvent *e)}\\
Metoda po kliknutí do zobrazovacího okna uoží bod do polygonu podle setAB.\\

\textbf{void setAB()}\\
Metoda nastaví, kam se budou ukládat vložené body.\\

\textbf{void clearAll();}\\
Metoda smaže vše ze zobrazovacího okna.\\

\textbf{void clearResults();}\\
Metoda smaže výsledky Boolovských operací ze zobrazovacího okna.\\

\textbf{void setRes(std::vector\textless std::vector\textless QPointFB> > result)}\\
Metoda nastaví vektor polygonů s výsledky Boolovských operací.\\

\textbf{void setA(std::vector\textless QPointFB\textgreater polA\_)}\\
Metoda nastaví polygonu A.\\

\textbf{void setB(std::vector\textless QPointFB\textgreater polB\_)}\\
Metoda nastaví polygonu B.\\

\textbf{std::vector\textless QPointFB\textgreater getA()}\\
Metoda vrátí polygonu A.\\

\textbf{std::vector\textless QPointFB\textgreater getB()}\\
Metoda vrátí polygonu B.\\


\subsubsection{QPointFB}
Nová třída odvozená od třídy QPointF.(double alfa, double beta, bool inters, TPointPolygon pos). Alfa je koeficient alfa, pokud bod leží na přímce A. 
Beta je koeficient beta, pokud bod leží na přímce B. Inters je true, pokud bod je průsečíkem dvou přímek. Pos nabývá hodnot INSIDE, OUTSIDE, ON vůči danému polygonu.\\

\textbf{double getAlfa()}\\
Metoda vrátí hodnotu alfa bodu třídy QPointFB.\\

\textbf{void setAlfa(double alfa\_)}\\
Metoda nastaví hodnotu alfa bodu třídy QPointFB.\\

\textbf{double getBeta()}\\
Metoda vrátí hodnotu beta bodu třídy QPointFB.\\

\textbf{void setBeta(double beta\_)}\\
Metoda nastaví hodnotu beta bodu třídy QPointFB.\\

\textbf{bool getInters()}\\
Metoda vrátí hodnotu inters bodu třídy QPointFB.\\

\textbf{void setInters(bool inters\_)}\\
Metoda nastaví hodnotu inters bodu třídy QPointFB.\\

\textbf{TPointPolygon getPosition()}\\
Metoda vrátí hodnotu pos bodu třídy QPointFB.\\

\textbf{void setPosition(TPointPolygon pos\_)}\\
Metoda nastaví hodnotu pos bodu třídy QPointFB.\\

\textbf{bool operator \textless (const QPointFB \&p)}\\
Přetížený operátor pro porovnání bodů třída QPointFB podle x.\\

\subsubsection{Widget}

\textbf{void on\_pushButton\_clicked()}\\
Po stisknutí tlačítka Polygon A/B, se nastaví kreslení polygonu A nebo B.\\

\textbf{void on\_pushButton\_2\_clicked()}\\
Po stisknutí tlačítka Boolean Operations, se vypočte a zobrazí výsledek Boolovské operace, dle výběru v combo boxu.\\

\textbf{void on\_pushButton\_3\_clicked()}\\
Po stisknutí tlačítka Clear results, se vymaže výsledek Boolovské operace.\\

\textbf{void on\_pushButton\_4\_clicked()}\\
Po stisknutí tlačítka Clear All, se vymaže vše co je v obrazovém okně.\\

\textbf{void on\_pushButton\_5\_clicked()}\\
Po stisknutí tlačítka Buffer, se vypočte a zobrazí buffer nad objekty v obrazovém okně.\\

\textbf{void on\_pushButton\_6\_clicked()}\\
Po stisknutí tlačítka Import A/B, lze naimportovat dva polygony pro testování aplikace.\\




\clearpage
\section{Závěr}
V rámci úlohy byla vytvořena aplikace pro výpočet Boolovských operací nad dvěma polygony. Polygony lze ručně naklikat v obrazovém okně, nebo naimportovat z textového souboru. Následně je možné vypočítat průnik, sjednocení, rozdíl A-B a rozdíl B-A polygonů. Výsledek se vykreslí v zobrazovací okně a je vyšrafován a ohraničen silnější linií, aby byl na první pohled jasně zřejmý.\\
\\
V této aplikaci bylo zapotřebí ošetřit i případy, kdy je výsledkem linie či bod. Pro testování těchto případů byly vytvořeny testovací textové soubory, na kterých si může uživatel sám vyzkoušet funkčnost. Přesto pro případ, kdy je výsledkem 0D entita - bod, tak není bod zřetelně vyznačen. Po spouštění aplikace a volbě jednotlivých operací pro tento případ je bod správně vyhodnocen, což lze vidět na ukázce aplikace, kde je ve frameworku Qt v konzolovém řádku po příkazu qDebug vypsán. \\


\section{Náměty na vylepšení}
\subsection{Import polygonů}
Načítání polygonů ze souboru by mohlo být oddělené. Uživatel by načetl polygony ze dvou různých textových souborů a přitom by si zvolil, kam chce načtený polygon uložit A/B.

\subsection{Buffer}
Bohužel operace buffer nebyla z důvodu nedostatku času zprovozněna, proto byla z aplikace vyřazena, aby v ní nezůstávaly nefunkční části. Přesto by bylo určitě vhodné buffer dokončit. 

\subsection{Vykreslení 0D výsledku}
Bylo by vhodné ošetřit, že v případě, že výsledkem je 0D entita (bod), tak bude graficky lépe zvýrazněn, aby byl uživateli na první pohled výsledek zřejmý.

\subsection{Přehlednost}
Pro přehlednost by bylo určitě vhodné, aby byly jednotlivé polygony popsány, zda se jedná o polygon A nebo B. Případně aby bylo i v aplikaci vidět, zda uživatel vykresluje polygon A nebo B. V případě, že je poté zvolena operace rozdílu, tak nemusí být uživateli na první pohled jasné, od kterého polygonu se který odečítá.



\clearpage
\section{Reference}

\begin{enumerate}
\item TSAGARIS, Antonis. Vector ilustration basics for Android developers [online][cit. 21. 12.2018]. \\
Dostupné z: https://hackernoon.com/vector-illustration-basics-for-android-developers-part-3-boolean-operations-8a0ced922030 \\

\item  BAYER, Tomáš. Množinové operace s polygony [online][cit. 3. 1. 2019]. \\
Dostupné z: https://web.natur.cuni.cz/~bayertom/images/courses/Adk/adk9.pdf  \\


\end{enumerate}
\end{document}



 